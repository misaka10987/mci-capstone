\documentclass[aspectratio=169]{beamer}

\usetheme{Berkeley}
\usefonttheme[onlymath]{serif}

\newcommand{\R}{\mathbb{R}}
\newcommand{\msg}[1]{\; \text{#1} \;}
\newcommand{\st}{\msg{s.t.}}

\title{Filters and Limits}
\subtitle{MCI Capstone Project: an Introduction to Application of Topology}
\author{Jiankun Su}
\institute{No.2 High School of East China Normal University}
\date{\today}

\AtBeginSection[]{
	\begin{frame}
		\frametitle{\insertsectionhead}
		\tableofcontents[currentsection, hideallsubsections]
	\end{frame}
}

\begin{document}

	\begin{frame}
		\titlepage
	\end{frame}

	\begin{frame}{Table of Contents}
		\tableofcontents
	\end{frame}

	\section{Introduction}
	
	\begin{frame}{Common Concepts in Calculus}
		
		In calculus, we often use limits to study a function's behaviour near a certain point,
		and continuity to study a function's behaviour as it changes through $\R$.
		
		\begin{definition}[$\epsilon$-$\delta$ Language of Limit]
			$$
			\lim_{x \to  x_0} f(x) = y_0 \iff
			\forall \epsilon \in \mathbb R^+, \exists \delta \in \mathbb R^+ \st
			\forall \left| x-x_0 \right| < \delta, \left| f(x) - y_0 \right| < \epsilon
			$$
		\end{definition}
		
		\begin{definition}[Continuity]
			$f: \mathbb R \to \R$ is said to be \textbf{continuous}, if and only if
			$$
			\forall x_0 \in \R, \lim_{x \to x_0} f(x) = f(x_0)
			$$
		\end{definition}
	\end{frame}
	
	\begin{frame}{Continuity in Topology}
		In topology, there is also a definition for continuity,
		which describes what transformations between topological spaces are considered to be ``smooth''.
		\begin{definition}[Continuous Functions between Topological Spaces]
			For topological space $X$ and $Y$, a function $f: X \to Y$ is said to be \textbf{continuous}, if and only if
			$$
			\forall \msg{open set} V \subseteq Y, f^{-1}(V) \msg{is open}
			$$
		\end{definition}
	\end{frame}

	\section{Filter}
	
	\begin{frame}{Filter}
		
		Filters are a way to generalize the definition of ``tends to'' from $\R$ to more generic topological spaces. In other words, we are going to use the inclusion relationship between sets to replace the partial order we use on $\R$.
		
		\begin{definition}[Filter on a Set]
			A collection $\mathcal F$ of subsets of $X$ is said to be a \textbf{filter} on $X$, if and only if
			\begin{itemize}
				\item $\emptyset \notin \mathcal F$, $X \in \mathcal F$
				\item $A \in \mathcal F, A \subseteq B \subseteq X \implies B \in \mathcal F$
				\item $A, B \in \mathcal F \implies A \cap B \in \mathcal F$
			\end{itemize}
		\end{definition}
		
		Intuitively speaking, a filter is like a series of fishnets,
		starting from the universal set, and sequencially blocking away some elements.
		
	\end{frame}
	
	\begin{frame}{Neighbourhood}
		
		Neighbourhood would be our way to define the concept of ``convergence to a certain point'', for generic topological spaces.
		
		\begin{definition}[Neighbourhood]
			For topological space $X$, $x \in X$,
			a set $U \subseteq X$ is said to be a \textbf{neighbourhood} of $x$, if and only if
			$$
			\exists \msg{open set} V \in X \st x \in V \subseteq U
			$$
		\end{definition}
		
		We usually use $\mathcal N_x$ to denote the collection of all neighbourhood, or \textbf{neighbourhood system}, of $x$. It can be shown from definition that a neighbourhood system is a filter.
		
	\end{frame}
	
	\begin{frame}{Limit and Convergence}
		
		\begin{definition}[Convergence]
			Let $X$ be topological space.
			A filter $\mathcal F$ on $X$ is said to converge to point $x$, if and only if
			$$
			\mathcal N_x \subseteq \mathcal F
			$$
		\end{definition}
		
		\begin{definition}[Limit]
			Let $f: X \to Y$ be mapping from topological space $X$ to $Y$.
			$\mathcal F$ be filter on $X$.
			It is said that $y \in Y$ is limit of $f$ with respect to $\mathcal F$, or
			$$
			\lim_\mathcal F f = y
			$$
			if and only if the image filter $f(\mathcal F)$ converges to $y$.
		\end{definition}
		
	\end{frame}
	
	\begin{frame}{Example on $\R$}
		
		Take a familiar subject in calculus, $\R$, for example and apply our new definition.
		
		\begin{definition}[Limit on $\R$]
			Under standard topology on $\R$, for $f: \R \to \R$,
			$$
			\lim_{x \to x_0} f(x) = y_0
			\iff f(\mathcal N_{x_0}) \msg{converges to} y_0
			$$
		\end{definition}
	\end{frame}

	\section{Consistency}
	
	\begin{frame}{Consistency with $\epsilon$-$\delta$}
		\begin{theorem}[Consistency of the Filter Definition with $\epsilon$-$\delta$ Language]
		Under standard topology on $\R$, for $f: \R \to \R$,
		$$
		f(\mathcal N_{x_0}) \msg{converges to} y_0
		\iff \forall \epsilon \in \mathbb R^+, \exists \delta \in \mathbb R^+ \st
		\forall \left| x-x_0 \right| < \delta, \left| f(x) - y_0 \right| < \epsilon
		$$
		\end{theorem}
		\begin{proof}
			As for filter $\implies$ $\epsilon$-$\delta$:
			
			$\forall \epsilon$, let $I_y = (y_0 - \epsilon, y_0 + \epsilon)$.
			
			By definition, $I_y \in \mathcal N_{y_0}$, therefore $I_y \in f(\mathcal F)$, and $U = f^{-1}(I_y) \in \mathcal N_{x_0}$.
			
			Since $U$ is open, $\exists \msg{open interval} I_x \st x_0 \in I_x \in U$.
			
			Let $I_x = (x_0 - \delta_1, x_0 + \delta_2)$, then $\delta = \min(\delta_1, \delta_2)$ would satisfy the $\epsilon$-$\delta$ definition.
		\end{proof}
	\end{frame}
	
	\begin{frame}{Consistency with $\epsilon$-$\delta$}
		\begin{proof}
			As for $\epsilon$-$\delta$ $\implies$ filter:
			
			$\mathcal N_{y_0} \subseteq f(\mathcal N_{x_0}) \iff \forall U \in \mathcal N_{y_0}, f^{-1}(U) \in \mathcal N_{x_0}$
			
			$\forall U$, since $U$ is open, $\exists I_y = (y_0 - \epsilon_1, y_0 + \epsilon_2)$ s.t. $y_0 \in I_y \subseteq U$.
			
			Let $\epsilon = \max(\epsilon_1, \epsilon_2)$. As the $\epsilon$-$\delta$ language suggests, $\exists \delta$ s.t. for $I_x = (x_0 - \delta, x_0 + \delta)$, $f(I_x) \subseteq (y_0 - \epsilon, y_0 + \epsilon) \subseteq I_y \subseteq U$.
			
			Therefore $I_x \subseteq f^{-1}(U)$. Since $I_x$ is a neighbourhood to $x_0$, $U \in \mathcal N_{x_0}$.
		\end{proof}
	\end{frame}

	\section{Application}
	
	\begin{frame}{}
		\begin{center}
			% \includegraphics[width=0.7\textwidth]{example-image}
			% \captionof{figure}{占位图说明}
		\end{center}
	\end{frame}

	\section{Conclusion}

	\begin{frame}{Thanks}
		\centering
		{\Huge \textbf{Thank You!}}\\[2em]
		{Feel free to contact me if you have any questions}\\[1em]
		{Email: misaka10987@outlook.com}\\
		{Wechat: misaka10987}\\
		{QQ: 2208129531}
	\end{frame}
	
\end{document}
